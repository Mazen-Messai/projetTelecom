\documentclass[12pt,a4paper]{article}

\usepackage[utf8]{inputenc}
\usepackage[T1]{fontenc}
\usepackage[french]{babel}
\usepackage{geometry}
\usepackage{tikz}
\usepackage{subcaption}
\usepackage{caption}
\usepackage{subcaption}
\usepackage{amsmath}
\usepackage{amssymb}


\geometry{margin=2cm}

\title{Projet de Télécommunication\\\large Introduction à l'égalisation}
\author{Elena Fleury, Mazen Messai}
\date{\today}

\begin{document}

\maketitle

\section{Étude théorique}
\textbf{Question 1 :}
Si on considère qu'il n'y aucun obstacle entre l'émetteur et le recepteur, on peut modéliser le signal reçu avant égalisation par la relation suivante :
\[
y_e(t) = \alpha_0 (t-\tau_0) + \alpha_1 x_e(t-\tau_1)
\]
Avec $\alpha_0$ le coefficient d'attenuation directe, $\alpha_1$ le coefficent d'attenueation réfléchie, et $\tau_0$ et $\tau_1$ les retards induits
\newline \par

\textbf{Question 2 :}
En écrivant $y_e = h_c(x_e)$ en déduit que $h_c(t)$ s'écrit :
\[
h_c(t) = \alpha_0 \delta(t-\tau_0) + \alpha_1 \delta(t-\tau_1)
\]
\newline \par

\textbf{Question 3 :}
On prend pour $h$ et $h_r$ une réponse en fenêtre réctangulaire de largeur $T_s$.
On a donc  $z(n) = h_r * h_c * h(n) = h_r * [\alpha_0 h(t-\tau_0) + \alpha_1 h(t-\tau_1)] = (\alpha_0 + \alpha_1)g(t) = 1.5 g(t)$
Avec $g(t)$ la réponse impulsionnelle du filtre de réception dont le tracé est donné par la figure 1.
\begin{center}
\begin{tikzpicture}[scale=2]
    % Réponse impulsionnelle triangulaire de hauteur T_s et largeur 2T_s
    \draw[->] (-0.5,0) -- (2.5,0) node[right] {$t$};
    \draw[->] (0,-0.2) -- (0,1.2) node[above] {$h_r*h(t)$};

    % Graduation sur l'axe des abscisses
    \draw[thick] (0,0.08) -- (0,-0.08);
    \node[below] at (-0.15,-0.08) {$0$};

    \foreach \x/\label in {1/{T_s}, 2/{2T_s}} {
        \draw[thick] (\x,0.08) -- (\x,-0.08);
        \node[below] at (\x,-0.08) {$\label$};
    }

    % Graduation sur l'axe des ordonnées
    \draw[thick] (0.08,1) -- (-0.08,1);
    \node[left] at (-0.08,1) {$T_s$};

    % Triangle
    \draw[thick,blue] (0,0) -- (1,1) -- (2,0);

    % Optionnel : pointillés pour visualiser la hauteur
    \draw[dashed,gray] (1,0) -- (1,1);
\end{tikzpicture}
\end{center}
\begin{center}\textit{Figure 1} - Réponse impulsionnelle du filtre de $h_r * h(t)$\end{center}


\begin{center}
\begin{tikzpicture}[scale=1.1]
    % Quadrillage
    \draw[very thin,gray!50] (0,-1.5) grid[xstep=1,ystep=0.5] (8,1.5);

    % Axes
    \draw[->] (0,-1.7) -- (0,1.7) node[above] {$z(n)$};
    \draw[->] (-0.3,0) -- (8.5,0) node[right] {$t$};

    % Graduation Ts sur l'axe des abscisses
    \draw[thick] (0,0.08) -- (0,-0.08);
    \node[below] at (-0.15,-0.08) {$0$};
    \foreach \x in {1,...,8} {
        \draw[thick] (\x,0.08) -- (\x,-0.08);
        \node[below] at (\x,-0.08) {$\x T_s$};
    }

    % Graduation sur l'axe des ordonnées
    \foreach \y in {-1.5,1.5} {
        \draw[thick] (0.08,\y) -- (-0.08,\y);
        \node[left] at (-0.08,\y) {$\y T_s$};
    }

    % Ajout des lignes
    \draw[thick,blue] (1,0) -- (2,1) -- (4,-1) -- (6,1) -- (7,0);
    \draw[thick,blue] (0,0) -- (1,-1) -- (3,1) -- (5,-1) -- (6,0);
    \draw[thick,red] (0,0) -- (1,-1) -- (2,1) -- (3,1)--(4,-1)--(5,-1) -- (6,1)--(7,0);
\end{tikzpicture}
\end{center}
\begin{center}\textit{Figure 2} - Signal en sortie du filtre de réception pour la séquence 011001 \end{center}
\newpage\par
\textbf{Question 4 :}
En prenant $t_0 \in \{0, 2T_s\}$, on à bien uniquement deux points sur le diagramme de l'oeil. Il est donc possible de respecter le critère de Nyquist sur cette chaîne de transmission.
\begin{center}
\begin{tikzpicture}[scale = 1.5]
    % Axes
    \draw[->] (-0.5,1) -- (2.2,1) node[right] {$t$};
    \draw[->] (0,-0.1) -- (0,2.1) node[above] {$T_s$};

    % Graduation sur l'axe des abscisses
    \draw[thick] (0,0.95) -- (0,1.05);
    \node[below] at (-0.15,0.90) {0};
    \foreach \x/\label in {1/{T_s}, 2/{2T_s}} {
        \draw[thick] (\x,0.95) -- (\x,1.05);
        \node[below] at (\x,0.90) {$\label$};
    }

    % Graduation sur l'axe des ordonnées

    \draw[thick,blue] (0,2) -- (2,2) -- (0,0) -- (2,0) -- (0,2) -- (1,0) -- (2,2) -- (0,0) -- (1,2) -- (2,0);
    \end{tikzpicture}
    \end{center}
\begin{center}\textit{Figure 3} - diagramme de l’oeil sans bruit en sortie du filtre de réception $h_r(t)$ \end{center}

\textbf{Question 5 :}
On prend $t_0 = T_s$, donc on à un seuil nul, le critère de Nyquist qui est respecté, et nous considérons que les symboles sont indépendants et équiprobables.
On a donc : $\displaystyle \text{TEB}= Q\bigg (\dfrac{V g(t_0)}{\sigma_w} \bigg) =Q\bigg (\dfrac{3T_s}{2\sigma_w} \bigg) $.
\vspace{0.5cm}
\par
\textbf{Question 6 :} $ \displaystyle\sigma^2_w = \int_{\mathbb{R}} N_0 | H_r(f)|^2 \,\mathrm{d}f = N_0 \int_{\mathbb{R}} |h_r(t)|^2 \,\mathrm{d}t = N_0 T_s$
\newline \par

\textbf{Question 7 :} $E_S = P_sT_s = \dfrac{3}{2}T_s^2$
\newline \par

\textbf{Question 8 :} On à $T_s = \sqrt{\dfrac{2}{3}E_s}$ et $E_b = E_s$ (mapping binaire), on  en déduit donc que : \newline
 $\displaystyle \text{TEB} =Q\bigg (\dfrac{3T_s}{2\sigma_w} \bigg) = Q\bigg(\sqrt{\dfrac{3}{2}\dfrac{E_s}{N_0}}\bigg)= Q\bigg(\sqrt{\dfrac{3}{2}\dfrac{E_b}{N_0}}\bigg)$ \newline
 

\end{document}